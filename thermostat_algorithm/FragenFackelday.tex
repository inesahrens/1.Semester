\documentclass[]{article}
\usepackage[utf8]{inputenc}
\usepackage{german}
\usepackage{amsmath}
\usepackage{amssymb}
\usepackage{amsthm}

\title{Fragen}

\begin{document}

\maketitle

\section{Temperature Relaxation by Stochastic Dynamics}

\subsection{Stochastic Forces}
Was sind stochastische Kräfte? Ich habe mir ein wenig zur Langevin Gleichung durchgelesen und habe dann den Zusammenhang zur Brownschen Bewegung angesehen. Sind die stochastischen Kräfte die zufälligen Zusammenstöße mit anderen Teilchen? 

\subsection{NVT ensemble}
Auf S.121 nach Formel 32 steht, dass die Tranjetorie, die durch die Integration der Langevin Bewegungsgleichung erzeugt wird, zu einer kanonischen Verteilung der Mikrozustände führt. Was ist die kanonische Verteilung der Mikrozustände? Ein kanonisches Ensemble? Warum führt die Integration der Langevin Bewegungsgleichung zu einer kanonischen Verteilung der Mikrozustände?



\section{Temperature Relaxation by Stochastic Coupling}

Kurze Einfuehrung: Wir wollen kollisisionen von Atomen mit einem Waermebad simulieren.
Dazu waehlen wir nach einer bestimmten Verteilung Atome aus, deren Geschwindigkeit dann nach der Maxwell-Boltzmann Verteilung neu gesetzt wird. 

\subsection{Waehlen wann ein Atom eine neue Geschwindigkeit bekommt}
Wir muessen bestimmen wann die Atome eine neue Temperatur bekommen sollen. Dazu gibt es zwei wege. Der erste ist nach einer bestimmten Wkeit, die $p(\tau)= \alpha \exp{- \alpha \tau}$ ist. Der zweite Weg soll einfacher sein. Ich verstehe nicht, warum dieser Weg einfacher sein soll. 

Bei Formel 39 wird die Wkeitsverteilung angegeben, wann ein Atom eine neue Geschwindigkeit bekommt. Die Herleitung der Wkeitsverteilung ist leider nicht angegeben. Wie komme ich auf diese Formel? 

Auch von Formel 41 gibt es keine Herleitung. Intuitiv sind mir einige Terme der Formel klar, jedoch nicht die genaue Zusammensetzung. Koennten Sie mir die Formel genauer erklaeren? 

\subsection{Mikrokaninisches vs kanonisches Ensemble}
so wie ich das Kapitel bis jetzt verstanden habe, wollen wir ein kanonisches Ensemble generieren, also eins, wo die Temperatur konstant ist. 


\subsection{Begriff: interne/externe Freiheitsgrade}
u.a. auf S.124 oben (und an anderen Stellen) tauchen die Begriffe interne und externe Freiheitsgrade auf. Was sind interne/ externe Freiheitsgrade? Haben Sie ein Beispiel?  

\subsection{Veraenderung der kinetischen Energie}
Im Text wird gesagt, dass jede Kollision die kinetische Energie des Systems um $\frac{3}{2} k_B (T_0-\mathcal{T})$ veraendert. Wie kommt man auf diesen Wert? 

\subsection{Formelzeichen}
Bei Formel 42 wird das Formelzeichen $c_{nu}$ benutzt. Wofuer steht $c_{nu}$ ?

\end{document}
