\documentclass[twocolumn]{article}
\usepackage[utf8]{inputenc}
\usepackage{german}
\usepackage{amsmath}
\usepackage{amssymb}
\usepackage{amsthm}

\let\tempone\itemize
\let\temptwo\enditemize
\renewenvironment{itemize}{\tempone\addtolength{\itemsep}{-.5\baselineskip}}{\temptwo}
\title{stichpkt}

\begin{document}

\section{Temperatur Relaxatierung mit stochastischer Dynamik}

\subsection{Das System}
\begin{itemize}
	\item abgeschlossenes System
	\item geloeste Moelkuele: 
	\begin{itemize}
		\item Loesungsmittel loest die Molekuele ohne chem. Reaktion
		\item durch loesungsmittel wird chemische Reaktion thermisch kontrolierbar
	\end{itemize}	
	\item was sind genau vakuum RB? %todo recherche? 
\end{itemize}

\subsection{klassische Molekueldynamik}
\begin{itemize}
	\item i Partikel
	\item x Koordinate 
	\item v Geschwindigkeit
	\item F systematisches Kraftfeld
	\item F kann durch Kraftfeld berechnet werden
	\item Kraftfeld von Interaktion der Partikel abhaengig
	\item jedes Molekuel Koord u. Bewegung $\rightarrow$ hochdim System 
	\item max System mit 50-1000 Partikeln simuliert $\Rightarrow $ troepchengroesse
	\item unerwünschte Randeffekte
	\item SD umgeht Problem
	\item bewegungsdetails von Loesemittel unwichtig $\Rightarrow$ keine exakte Simulation
	\item werden durch stocha Kraft beschrieben, die auf anderen Partikel wirkt	
	\item resultiert folgende formel
\end{itemize}

\subsection{Langevin Gleichung}
\begin{itemize}
  	\item Reibungskoeffizient $\gamma_i$ abhaengig von der Viskositaet des Loesungsmittels
  	\item Die stochastische Kraft $R_i(t)$ beschreibt kollisionen mit Partikeln des Loesungsmittels
  	\item umgebendes Loesungsmittel ist in systematischer Kraft $F_i$ eingebunden
\end{itemize}

\subsection{Stochastische Kraft $R_i$}
\begin{itemize}
	\item stationäre Gaussche ZV
	\begin{itemize}
		\item Kollisionen mit Partikel als ZV darstellen
		\item Kollisionen mit Partikeln sind zeitinvariant $\Rightarrow$ stationaer
		\item normalverteilt modelliert 
	\end{itemize}
	\item Mittelwert ueber Zeit ist Null
	\begin{itemize}
		\item Kollisionen kommen von allen Seiten gleich oft vor im Mittel
	\end{itemize}
	\item kein Zusammenhang zu vorherigen Geschwindigkeiten oder der systematischen Kraft.
	\begin{itemize}
		\item kollisionen unabhaengig davon wie schnell die anderen Partikel sind oder welche Kraft auf diese wirkt
	\end{itemize}	 
	\item der quadratische Mittelwert von $R_i$ berechnet sich zu $2 m_i \gamma_i k_B T_0$ %todo bedeutung der Formel
	\begin{itemize}
		\item $\gamma_i$ Reibungskoeffizient
	\end{itemize}
	\item die $R_{i \mu}$ sind unabhängig voneinander
	\begin{itemize}
		\item $\mu$ x,y od. z Achse
		\item betrachte $R_{i \mu}$, $R_{j \nu}$ verschieden. Voneinander unabhaengig 
	\end{itemize}
	\item zusammenfassung der letzten beiden%todo do not understand
\end{itemize}



\subsection{Reibungskoeefizient $\gamma_i$} 
\begin{itemize}
	\item 0: siehe Formel: $\gamma_i=0$ ein Teil der Formel faellt weg. letzter Teil faellt weg  $\Rightarrow$ Newtonsche Bewegungsgl. MD %todo warum faellt letzter Teil weg? (nach quadratischen mittel)
	\item zu klein: schlechte Temperaturregelung, kanonisches Ensemble wird erst spaet erreicht, anhaufung von numerischen Fehlern, falsch simuliert
	\item zu gross: stoerrt Dynamik des Systems 
\end{itemize}

\subsection{Reibungskoeefizient $\gamma_i$} 
\begin{itemize}
	\item Ziel:Wert für Reibungskoeffizienten festsetzen
\end{itemize}

\subsection{Reibungskoeefizient $\gamma_i$: Herleitung} 
\begin{itemize}
	\item . $\Delta \tau$ ist ein Zeitintervall, Veränderung der Temperatur  wird beobachtet
	\item $\dot{r}_i$ geschwindigkeit des iten Teilchens
\end{itemize}

\subsection{Reibungskoeefizient $\gamma_i$: Herleitung} 
wie von franziska gezeigt wurde

\subsection{Eigenschaften der stochastischen Dynamik}
\begin{itemize}
	\item Phasenraum: jeder Pkt ist bestimmter Zustand des Systems, jeder pkt beschreibt zu jedem simulierten teilchen alle Eigenschaften. Betrachte nun Trajektorie. 
	\item Trajektorie verfügbar und stetig 
	\begin{itemize}
		\item verfügbar: können durch die Bewegungsgleichung Trajektorie nachvollziehen
		\item stetig: %todo warum
	\end{itemize}
	\item Trajektorie nicht deterministisch 
	\begin{itemize}
		\item Deterministisch: durch Vorbedingungen eindeutig festgelegt. hier nicht, da stochastische Variablen, die sich bei jedem mal verändern können. Prozess zweimal ausführen $->$ unterschiedliche Ergebnisse
	\end{itemize}
	\item Bewegungsgleichung nicht zeitreversibel 
	\begin{itemize}
		\item Prozess kann umgekehrt werden, ohne dass Veränderungen im System stattfinden. hier nt mgl, da stochastische Terme beim Umkehren etwas anders aussehen -> anderer Endzustand als vorher. 
	\end{itemize}	
\end{itemize}



\section{Temperatur Relaxatierung mit stochastischer Verknüpfung}

\subsection{Das System}
	\begin{itemize}
		\item geschlossenes System: kein Partikelaustausch:  konst. Volumen, konst. \# Partikel
		\item Wärmeaustausch mit Wärmebad
		\item Ziel: System bei konstanter Temperatur simulieren
		\item Wie Wärmebad simulieren? Anderson Thermostat
	\end{itemize}

\subsection{Idee des Anderson Thermostats}
	\begin{itemize}
		\item Partikel kollidieren mit Wärmebad
		\item Kollisionen durch zufällige stochastische Kraft simuliert, die auf Partikel wirkt
		\item bei Kollision neue Geschwindigkeit für Teilchen
		\item kinetische Energie verändert sich
		\item Umsetzung: Newtonsche Bewegungsgleichung
		\item bei jeder Kollision gestört
		\item Zu welchem Zeitpkt kollidieren Teilchen?
		\item Wie sehen neue Geschwindigkeiten aus? 
	\end{itemize}

\subsection{Zeitpunkt der Kollision}
	\begin{itemize}
		\item betrachte zuerst Zeitpunkt der Kollision; wann neue Geschwindigkeit
		\item Betrachte nur eine Teichen i
		\item Zeitintervall $\tau$ zwischen zwei aufeinanderfolgenden Kollisionen
		\item geg durch Wkeitsverteilung $p(\tau)= \alpha e^{- \alpha \tau}$
		\item $\alpha$ Kollisionsfrequenz
		\item  vor Simulation Festlegung zufälliger Folge von Zeitintervallen für Geschwindigkeitsneuzuordnung 
	\end{itemize}

\subsection{Wahl der neuen Geschwindigkeit}
	\begin{itemize}
		\item Geschwindigkeit des Partikels ändert sich in jeder Koordinate gemäß einer Maxwell-Boltzmann Verteilung
		\item i Teilchen, $\mu$ Koordinatenache, $r$ Position, $T_0$ Referenztemperatur, wollen System auf diese Temperatur bringen, $m_i$ Masse 
		\item  Maxwell-Boltzmann-Verteilung beschreibt  statistische Verteilung des Betrages der Teilchengeschwindigkeit im Idealen Gas %todo genauer? 
	\end{itemize}
	

\subsection{Newtonsche Bewegungsgleichung für das Anderson Thermostat}
	\begin{itemize}
		\item Newtonsche Bewegungsgleichung für Anderson Thermostat
		\item Grundlage: Newtonsche Bewegungsgleichung 
		\item n: Intervalle für die Neuzuweisungen
		\item $\delta$ Dirac Delta: 1, falls term in Klammer 0, 0 sonst. 
		\item $\dot{r}_{i,n}(t) $  neue Geschwindigkeit nach dem n-ten Intervall
		\item t nicht ende/ anfang des Intervalls, normale Bewegungsgl.
		\item t anfangs/endzeitpkt von irgendeinem Intervall: addieren zu alter Bewegungsgleichung abstand von neuer zu alter Geschwindigkeit.  
	\end{itemize}
	
\subsection{Wahl der Kollisionsfrequenz $\alpha$}
	\begin{itemize}
		\item vorhin, bei Wahl der Intervalle $\tau$, Kollisionsfrequenz $\alpha$. Wie wählen?  
		\item gleiche wie bei stochastischer Dynamik
		\item Kollisionsfrequenz 0: keine Kollision mit Wärmebad, keine Veränderung, Molekulare Simulation
		\item  $\alpha $ zu klein: sehr selten Neuzuweisung von Geschwindigkeiten d.h. schlechte Temperaturkontrolle. 
		\item $\alpha$ zu groß: Dynamik des Systems gestört.
		\item kann gezeigt werden: enge Verbindung zwischen Temperatur Relaxationszeit $ \zeta_T$ und Kollisionsfrequenz besteht
		\item Temperatur Relaxationszeit: Zeit, in dem sich das System dem stationären Zustand, hier: gleiche Temperatur, annähert.  
		\item N: Teilchenzahl, $c_\nu$ isochore Wärmekapazität
		\item Kollisionsfrequenz skaliert für jeden Partikel mit $N^{-2/3}$
		\item \# Kollisionen pro Partikel weniger, je größer das System
		\item (Rem: Wärmekapazität: Verhältnis zwischen zugeführter Wäre und dadurch resultierende Temperaturerhöhung) 
		\item (isochor: Zustandsänderung, bei der Volumen gleich bleibt)
	\end{itemize}
	

\subsection{Eigenschaften der stochastischen Verknüpfung}
	\begin{itemize}
		\item Trajektorie Verfügbar, stetig
		\begin{itemize}
			\item verfügbar: können durch die Bewegungsgleichung Trajektorie nachvollziehen
			\item stetig: %todo warum? 
		\end{itemize}
		\item Trajektorie nicht deterministisch
		\begin{itemize}
			\item nicht durch Vorbedingung eindeutig festgelegt, da Wahl der Intervalle sich ändert und auch Neuzuweisung der Geschwindigkeiten sich verändern kann
		\end{itemize}
		\item Bewegungsgleichung nicht Zeitreversibel 
		\begin{itemize}
			\item gleiche begründung wie bei nicht deterministisch: Stochastische Komponenten verhindern
		\end{itemize}		
	\end{itemize}
	
\end{document}