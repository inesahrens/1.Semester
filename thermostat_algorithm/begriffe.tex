\documentclass[]{article}
\usepackage[utf8]{inputenc}
\usepackage{german}
\usepackage{amsmath}
\usepackage{amssymb}
\usepackage{amsthm}

%opening
\title{wichtige begriffe der Thermodynamik}
\author{Ines Ahrens}

\begin{document}

\maketitle



\section{Begriffe}

Die Kraft ist die Ableitung des Potentials.
Hamiltonglg beschreibt die Energie der Zustände im Phasenraum abhängig von Impuls- und Ortskoordinaten.

$https://www.univie.ac.at/physikwiki/images/8/80/Der_Phasenraum.pdf$

\subsection{Feld}
Referenz: Wiki: Feld(Physik)

Ein Feld beschreibt die räumliche Verteilung einer physikalischen Größe. Beispielsweise kann die räumliche Verteilung der Temperatur einer Herdplatte durch ein Temperaturfeld beschrieben werden oder die räumliche Verteilung der Dichte in einem Körper durch ein Massendichtefeld.

\subsection{Fluss}
Referenz: Wiki: Fluss(Physik) 

Als Fluss werden verschiedene physikalische Größen bezeichnet, die sich als Produkt eines Feldes und einer Fläche ergeben. Beispiele sind magnetische Fluss, elektrischer Fluss, Volumenstrom. 

\subsection{Freiheitsgrad}
Referenz: Wiki: Freiheitsgrad

Als Freiheitsgrad F bzw. f wird die Zahl der voneinander unabhängigen (und in diesem Sinne „frei wählbaren“) Bewegungsmöglichkeiten eines Systems bezeichnet. Die einzelnen Bewegungsmöglichkeiten werden auch Freiheiten genannt. Ein starrer Körper im Raum hat demnach den Freiheitsgrad f = 6, denn man kann den Körper in drei voneinander unabhängige Richtungen bewegen (Translation) und um drei voneinander unabhängige Achsen drehen (Rotation).

Jedes Molekül mit n Atomen hat allgemein $f = 3n$ Freiheitsgrade, weil man für jedes Atom drei Koordinaten braucht, um seine Position zu definieren. Diese kann man formal in Translations-, Rotations- und innere Schwingungsfreiheitsgrade einteilen:

\begin{align} f & = f_\mathrm{trans} + f_\mathrm{rot} + f_\mathrm{vib}\\ 
\Rightarrow f_\mathrm{vib} & = 3n - f_\mathrm{trans} - f_\mathrm{rot} 
\end{align}

Komplexe Moleküle mit vielen Atomen haben daher viele Schwingungsfreiheitsgrade (siehe Molekülschwingung) und liefern somit einen hohen Beitrag zur Entropie.





\subsection{makrozustand}
Referenz: Wiki: Makrozustand

Ein Makrozustand beschreibt in der Thermodynamik und statistischen Physik ein System mit vielen Freiheitsgraden (also z. B. ein Gas, das aus 1 mol ca. $10^{23}$ Einzelteilchen besteht) durch einige wenige Zustandsvariablen, wie Energie, Temperatur, Volumen, Druck, chemische Zusammensetzung oder Magnetisierung.

In der Mechanik lässt sich ein System aus N Teilchen vollständig beschreiben, indem man jedem Teilchen einen Orts- und Geschwindigkeitsvektor zuordnet. Man spricht hier von einem Mikrozustand. Dieser kann durch einen Punkt im Phasenraum dargestellt werden.

Für viele Teilchen (Teilchenzahl$ N \sim 10^{23}$) ist es jedoch praktisch unmöglich, einen mikroskopischen Anfangszustand zu bestimmen oder die Bewegungsgleichung für das System zu lösen. In chaotischen Systemen ist die Bestimmung der Bahn des Systems auch prinzipiell unmöglich, da kleinste Änderungen der Anfangsbedingungen zu beliebig großen Abweichungen führen.

Die mikroskopische Lösung der Bewegungsgleichung ist aber auch gar nicht notwendig, da die makroskopischen Eigenschaften nur von wenigen Parametern abhängen.

Zu einem bestimmten Makrozustand, der durch wenige makroskopischen Zustandsvariablen festgelegt ist, sind sehr viele Mikrozustände möglich. Diese bilden eine kontinuierlich verteilte Gesamtheit im Phasenraum. Der Makrozustand wird also durch ein statistisches Konzept bestimmt (Wahrscheinlichkeitsverteilung der Mikrozustände). Die Schwankungen der makroskopischen Größen werden aufgrund der hohen Teilchenzahlen vernachlässigbar gering.

Mit makroskopischen Größen kann man so makroskopische, deterministische Gesetze aufstellen. Kennt man z.B. für ein Gas die makroskopischen Zustandsgrößen Volumen, Temperatur und Teilchenzahl, so lässt sich der Druck eindeutig berechnen (Thermische Zustandsgleichung idealer Gase).

\subsection{Thermodynamik}
Referenz: Wiki: Thermodynamik

Die Thermodynamik beschäftigt sich mit der Möglichkeit, durch Umverteilen von Energie zwischen ihren verschiedenen Erscheinungsformen Arbeit zu verrichten. Die Grundlagen der Thermodynamik wurden aus dem Studium der Volumen-, Druck-, Temperaturverhältnisse bei Dampfmaschinen entwickelt.

Man unterscheidet zwischen offenen, geschlossenen und abgeschlossenen (isolierten) thermodynamischen Systemen. Bei einem offenen System fließt – im Gegensatz zum geschlossenen – Materie über die Systemgrenze, abgeschlossene Systeme sind auch Energiedicht. Nach dem Energieerhaltungssatz bleibt darin die Summe aller Energieformen (thermische, chemische, Federspannung, Magnetisierung usw.) konstant.

Die Thermodynamik bringt die Prozessgrößen Wärme und Arbeit an der Systemgrenze mit den Zustandsgrößen in Zusammenhang, welche den Zustand des Systems beschreiben. Dabei wird zwischen intensiven Zustandsgrößen (beispielsweise Temperatur T, Druck p, Konzentration n und chemisches Potential $\mu$) und extensiven Zustandsgrößen (beispielsweise innere Energie U, Entropie S, Volumen V und Teilchenzahl N) unterschieden.

Auf der Basis von vier fundamentalen Hauptsätzen sowie materialspezifischen, empirischen Zustandsgleichungen zwischen den Zustandsgrößen (siehe z. B. Gasgesetz) erlaubt die Thermodynamik durch die Aufstellung von Gleichgewichtsbedingungen Aussagen darüber, welche Änderungen an einem System möglich sind (beispielsweise welche chemischen Reaktionen oder Phasenübergänge ablaufen können, aber nicht wie) und welche Werte der intensiven Zustandsgrößen dafür erforderlich sind. Sie dient zur Berechnung von frei werdender Wärmeenergie, von Druck-, Temperatur- oder Volumenänderungen, und hat daher große Bedeutung für das Verständnis und die Planung von Prozessen in Chemieanlagen, bei Wärmekraftmaschinen sowie in der Heizungs- und Klimatechnik.

\subsection{Thermodynamisches System}
Referenz: Wiki: Thermodynamisches System

Ein thermodynamisches System ist ein räumlich eingegrenzt betrachtetes physikalisches System, für das eine Bilanzgleichung wie eine Energiebilanz oder, bei offenen Systemen, eine Stoffbilanz aufgestellt werden kann. Beim geschlossenen System werden nur die Energien (Wärme und Arbeit) betrachtet, die über die Systemgrenze fließen und dadurch mit der Änderung der inneren Energie den Zustand des Systems verändern. Bei einem isolierten System findet keinerlei Austausch mit der Umgebung statt.


\subsection{thermodynamisches Gleichgewicht}
Referenz: Wiki: Gleichgewicht(Physik $->$ Thermodynamik) 

Allgemein ist ein Gleichgewicht die Ausgeglichenheit aller Potentiale und Flüsse in einem System.
 
Ein System ist im thermodynamischen Gleichgewicht, wenn es in einem stationären Zustand ist, in dem alle makroskopischen Flüsse verschwinden. Dann gilt grundsätzlich das Kräftegleichgewicht aus Gibbs freier Enthalpie. Das heißt, dass die Thermodynamischen Potenziale kein Gefälle haben, die eine Änderung der Potentialgrößen des Systems antreiben.

\subsection{Kraftfeld}
Referenz: Wiki: Kraftfeld(Computerphysik)

In der Computerphysik und verwandten Disziplinen ist ein Kraftfeld (englisch: force field) eine Parametrisierung der potentiellen Energie. Wenn auf ein bestimmtes Kraftfeld verwiesen wird, so wird sowohl auf die funktionelle Form des Kraftfeldes, als auch auf einen speziellen (festgelegten) Parameterset verwiesen.

Häufig enthalten Kraftfelder Terme für Beiträge zur potentiellen Energie, die durch chemische Bindungen vermittelt werden sowie Terme für Wechselwirkungen, die nicht durch chemische Bindungen vermittelt werden:

$E=E_\text{bonded}+E_\text{nonbonded}$.

Der Beitrag $E_\text{nonbonded}$ enthält häufig ein Lennard-Jones-Potential-Term und einen Coulomb-Potential-Term. Der Beitrag $E_\text{bonded}$ enthält häufig Terme, welche die Torsion von Bindungen, Bindungswinkel und Bindungslängen beschreiben.

Der Term der die Bindungslänge zwischen Atomtypen der Sorte A und B beschreibt, kann z.B. die Form $E_{AB} = k_{AB} (r_{AB}^0 - r)^2$ annehmen, wobei die Federkonstante $k_{AB}$ sowie der Gleichgewichtsabstand $r_{AB}^0$ Parameter sind. Da beispielsweise Kohlenstoffatome je nach dem ob eine Doppel- oder Einfachbindung vorliegt andere Gleichgewichtsabstände und Federkonstanten haben, verwendet man zur Charakterisierung der anzuwendenden Parameter nicht lediglich Elementsymbole, sondern Atomtypen. Bei der (alleinigen) Wahl der obigen funktionellen Form zur Beschreibung der Bindungslänge wäre das Brechen von Bindungen nicht möglich. Es gibt jedoch reaktive Kraftfelder (wie beispielsweise ReaxFF), die das Brechen von Bindungen beschreiben können.

Die Wahl der Parameter eines Kraftfeldes erfolgt so, dass es in Computersimulationen bestimmte Aspekte möglichst exakt wiedergeben kann.
 

\subsection{Moleküldynamik}
Referenz: Wiki Moleküldynamik-Simulation

Moleküldynamik oder Molekulardynamik (MD) bezeichnet Computersimulationen in der molekularen Modellierung, bei denen Wechselwirkungen zwischen Atomen und Molekülen und deren sich daraus ergebende räumliche Bewegungen iterativ berechnet und dargestellt werden. Bei der Modellierung von komplexen Systemen mit einer Vielzahl an beteiligten Atomen werden hauptsächlich Kraftfelder oder semiempirische Methoden verwendet, da der Rechenaufwand zur Anwendung von quantenmechanischen Verfahren (ab-initio-Methoden) hierbei zu groß wäre.

Die MD-Methode spielt eine große Rolle in der Simulation von Flüssigkeiten, wie z. B. Wasser oder wässrigen Lösungen, wo strukturelle und dynamische Eigenschaften in experimentell schwer zugänglichen Bereichen (z. B. von Druck und Temperatur) berechnet werden können.

Aus Sicht der statistischen Physik erzeugt eine MD-Simulation Konfigurationen, die bestimmten thermodynamischen Ensembles entsprechen.  Monte-Carlo-Simulationen erzeugen vergleichbare Konfigurationen unter Verwendung der Zustandssumme dieser Ensembles.

Das simulierte Volumenelement wird am Anfang mit den zu untersuchenden Teilchen gefüllt. Anschließend folgt die Equilibrierung: es werden für jedes Teilchen die Kräfte berechnet, die auf es aufgrund seiner Nachbarn wirken, und die Teilchen entsprechend dieser Kräfte in sehr kleinen Zeitschritten bewegt. Nach einigen Schritten (bei einem guten, passenden Kraftmodell) gelangt das Probevolumen in ein thermisches Gleichgewicht, und die Teilchen fangen an, sich "sinnvoll" zu bewegen. Nun können aus den Kräften und Bewegungen der Teilchen Druck und Temperatur berechnet und schrittweise verändert werden.

MD-Simulationen finden meist unter periodischen Randbedingungen statt: jedes Teilchen, das das simulierte Volumen auf einer Seite verlässt, taucht auf der gegenüberliegenden wieder auf, alle Wechselwirkungen finden auch über diese Grenzen hinweg direkt statt. Dazu werden identische Kopien des simulierten Volumens nebeneinandergesetzt, so dass der dreidimensionale Raum die Oberfläche eines flachen, vierdimensionalen Torus bildet. Da dabei zu jedem Teilchen in den benachbarten Zellen (3x3x3-1=) 26 Kopien entstehen, werden kurzreichweitige Wechselwirkungen nur zu dem einen, nächstliegenden dieser identischen Bildteilchen berechnet .




\end{document}
