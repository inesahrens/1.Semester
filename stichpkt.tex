\documentclass[]{article}
\usepackage[utf8]{inputenc}
\usepackage{german}
\usepackage{amsmath}
\usepackage{amssymb}
\usepackage{amsthm}

\title{stichpkt}

\begin{document}

\section{Temperatur Relaxatierung mit stochastischer Dynamik}

\subsection{Das System}
\begin{itemize}
	\item abgeschlossenes System
	\item geloeste Moelkuele: 
	\begin{itemize}
		\item Loesungsmittel loest die Molekuele ohne chem. Reaktion
		\item durch loesungsmittel wird chemische Reaktion thermisch kontrolierbar
	\end{itemize}	
	\item was sind genau vakuum RB? %todo recherche? 
\end{itemize}

\subsection{klassische Molekueldynamik}
\begin{itemize}
	\item i Partikel
	\item x Koordinate 
	\item v Geschwindigkeit
	\item F systematisches Kraftfeld
	\item F kann durch Kraftfeld berechnet werden
	\item Kraftfeld von Interaktion der Partikel abhaengig
	\item jedes Molekuel Koord u. Bewegung $\rightarrow$ hochdim System 
	\item max System mit 50-1000 Partikeln simuliert $\Rightarrow $ troepchengroesse
	\item unerwünschte Randeffekte
	\item SD umgeht Problem
	\item bewegungsdetails von Loesemittel unwichtig $\Rightarrow$ keine exakte Simulation
	\item werden durch stocha Kraft beschrieben, die auf anderen Partikel wirkt	
	\item resultiert folgende formel
\end{itemize}

\subsection{Langevin Gleichung}
\begin{itemize}
  	\item Reibungskoeffizient $\gamma_i$ abhaengig von der Viskositaet des Loesungsmittels
  	\item Die stochastische Kraft $R_i(t)$ beschreibt kollisionen mit Partikeln des Loesungsmittels
  	\item umgebendes Loesungsmittel ist in systematischer Kraft $F_i$ eingebunden
\end{itemize}

\subsection{Stochastische Kraft $R_i$}
\begin{itemize}
	\item stationäre Gaussche ZV
	\begin{itemize}
		\item Kollisionen mit Partikel als ZV darstellen
		\item Kollisionen mit Partikeln sind zeitinvariant $\Rightarrow$ stationaer
		\item normalverteilt modelliert 
	\end{itemize}
	\item Mittelwert ueber Zeit ist Null
	\begin{itemize}
		\item Kollisionen kommen von allen Seiten gleich oft vor im Mittel
	\end{itemize}
	\item kein Zusammenhang zu vorherigen Geschwindigkeiten oder der systematischen Kraft.
	\begin{itemize}
		\item kollisionen unabhaengig davon wie schnell die anderen Partikel sind oder welche Kraft auf diese wirkt
	\end{itemize}	 
	\item der quadratische Mittelwert von $R_i$ berechnet sich zu $2 m_i \gamma_i k_B T_0$ %todo bedeutung der Formel
	\begin{itemize}
		\item $\gamma_i$ Reibungskoeffizient
	\end{itemize}
	\item die $R_{i \mu}$ sind unabhängig voneinander
	\begin{itemize}
		\item $\mu$ x,y od. z Achse
		\item betrachte $R_{i \mu}$, $R_{j \nu}$ verschieden. Voneinander unabhaengig 
	\end{itemize}
	\item zusammenfassung der letzten beiden%todo do not understand
\end{itemize}



\subsection{Reibungskoeefizient $\gamma_i$} 
\begin{itemize}
	\item 0: siehe Formel: $\gamma_i=0$ ein Teil der Formel faellt weg. letzter Teil faellt weg  $\Rightarrow$ Newtonsche Bewegungsgl. MD %todo warum faellt letzter Teil weg? (nach quadratischen mittel)
	\item zu klein: schlechte Temperaturregelung, kanonisches Ensemble wird erst spaet erreicht, anhaufung von numerischen Fehlern, falsch simuliert
	\item zu gross: stoerrt Dynamik des Systems 
\end{itemize}

\subsection{Reibungskoeefizient $\gamma_i$} 
\begin{itemize}
	\item Ziel:Wert für Reibungskoeffizienten festsetzen
\end{itemize}

\subsection{Reibungskoeefizient $\gamma_i$: Herleitung} 
\begin{itemize}
	\item . $\Delta \tau$ ist ein Zeitintervall, Veränderung der Temperatur  wird beobachtet
	\item $\dot{r}_i$ geschwindigkeit des iten Teilchens
\end{itemize}

\subsection{Reibungskoeefizient $\gamma_i$: Herleitung} 
wie von franziska gezeigt wurde

\subsection{Eigenschaften der stochastischen Dynamik}
\begin{itemize}
	\item Phasenraum: jeder Pkt ist bestimmter Zustand des Systems, jeder pkt beschreibt zu jedem simulierten teilchen alle Eigenschaften. Betrachte nun Trajektorie. 
	\item Trajektorie verfügbar und stetig %todo warum
	\item Trajektorie nicht deterministisch   %todo warum
	\item Bewegungsgleichung nicht Zeitreversibel %todo warum? 
\end{itemize}

%Ein reversibler Prozess ist eine thermodynamische Zustandsänderung von Körpern, die jederzeit wieder umgekehrt ablaufen könnte, ohne dass die Körper oder deren Umgebung dabei bleibende Veränderungen erfahren.









\section{Temperatur Relaxatierung mit stochastischer Verknüpfung}

\subsection{Das System}

\subsection{Idee des Anderson Thermostats}

\subsection{Umsetzung des Anderson Thermostats}

\subsection{Zeitpunkt der Kollision}

\subsection{Wahl der neuen Geschwindigkeit}

\subsection{Newtonsche Bewegungsgleichung für das Anderson Thermostat}

\subsection{Wahl der Kollisionsfrequenz $\alpha$}

\subsection{Eigenschaften der stochastischen Verknüpfung}


\end{document}