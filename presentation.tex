\documentclass{beamer}
\usetheme{Warsaw}  %% Themenwahl
\usepackage[utf8]{inputenc}
\usepackage{german}
\usepackage{amsmath}
\usepackage{amssymb}
\usepackage{amsthm}

\title{Thermostat Algorithmen}
\author{Franziska Engbers, Ines Ahrens}
\date{\today}

\begin{document}
\maketitle
\frame{\tableofcontents[currentsection]}

\section{Temperatur Relaxatierung mit stochastischer Dynamik}

\begin{frame}
	\frametitle{Das System}
	\begin{itemize}
		\item konstantes Volumen
		\item konstante Anzahl an Partikeln	
		\item betrachte Molekuele in einer Loesung
		\item Vakuum Randbedingungen
		\item Ziel: System bei konstanter Temperatur modellieren		
	\end{itemize}
	skizze?
\end{frame}

\begin{frame} 
  \frametitle{klassische Molekueldynamik}
  \begin{block}{Newtonsche Bewegungsgleichung} 
	\begin{align*}
	\dot{x}_i(t) = v_i(t) \\
	m_i(t) \dot{v}_i(t) = F_i(\{ x_i(t)\}) 
	\end{align*}
  \end{block}
  \begin{itemize}
  	\item jedes Molekuel wird simuliert
  	\item nur Troepfchengroesse kann simuliert werden 
  	\item unerwuenschte Randeffekte
  \end{itemize}
  \end{frame}
  
  \begin{frame}
  \frametitle{Langevin Gleichung}
%todo kein space freilassen vor formel!
  \begin{block}{Langevin Gleichung} %%Definition
	\begin{align*}
	\dot{v}_i(t)  = m_i^{-1} F_i(\{x_i(t)\}) - \gamma_i v_i(t) + m_i^{-1} R_i(t)
	\end{align*}
  \end{block}  
  \begin{itemize}
  	\item $\gamma_i$: Reibungskoeffizient 
  	\item $R_i(t)$: stochastische Kraft  
  \end{itemize}
  
\end{frame}

\begin{frame}
  \frametitle{Stochastische Kraft $R_i$}
  \begin{block}{Eigenschaften der stochastischen Kraft} 
  	\begin{enumerate}
  		\item stationäre Gausssche Zufallsvariable
  		\item Null Zeitmittelwert %todo warum
  		\item keine Korrelation zu vorherigen Geschwindigkeiten oder der systematischen Kraft. 
  		\item quadratischer Mittelwert: $2 m_i \gamma_i k_B T_0$
  		\item die $R_{i \mu}$ sind zueinander unabhängig
  	\end{enumerate}
  \end{block}	
  
    \visible<2>{  4 und 5 lassen sich zusammenfassen:
    	\begin{align*}
	    	\langle R_{i \mu} R_{j \nu } \rangle = 2 m_i \gamma_i k_B T_0 \delta_{ij} \delta_{\mu \nu} \delta(t' - t)
    	\end{align*}}
  
\end{frame}


\begin{frame}
	\frametitle{Reibungskoeefizient $\gamma_i$}
	Wiederholung: Langevin Gleichung:
	\begin{align*}
		\dot{v}_i(t)  = m_i^{-1} F_i(\{x_i(t)\}) - \gamma_i v_i(t) + m_i^{-1} R_i(t)
	\end{align*}
	\hspace{2em}
	
	\begin{description}
		\item[$\gamma_i=0 \forall i$:] Molekulare Simulation
		\item[$\gamma_i$ zu klein:] kaum Temperaturkontrolle
		\item[$\gamma_i$ zu groß:] Dynamik des Systems gestört 
	\end{description}
\end{frame}


\begin{frame}
	\frametitle{Reibungskoeefizient $\gamma_i$: Herleitung}
	\begin{itemize}
	    \item Ansatz: $\gamma_i$ konstant $\gamma$ für alle Partikel
	    \item $\gamma$ groß im Vergleich zur Beschleunigung
	    \item Langevin Gleichung vereinfacht sich zu:
	    \begin{align*}
	    v_i(t) = \gamma^{-1} m_i^{-1} (F_i(t) + R_i(t) )
	    \end{align*}
	    \item durch Definition der Temperatur, der kinetischen Energie und der Eigenschaften der stochastischen Kraft ergibt sich: 
	     
	\end{itemize}
    	
\end{frame}

%todo erst später unter der Klammer sichtbar? 
\begin{frame}
	\frametitle{Reibungskoeefizient $\gamma_i$: Herleitung}
	\begin{align*}
		\frac{\Delta \mathcal{T}}{\Delta \tau} = \frac{2}{k_B N_{df}} \sum\limits_{i=1}^N  \underbrace{\overline{F_i \dot{r}_i} }_ {\substack{\text{ Veränderung der Temperatur} \\ \text{ durch systematische Kraft}} }+\underbrace{ 2 \gamma (T_0 - \overline{\mathcal{T}})  }_{\substack{\text{Veränderung der Temperatur} \\ \text{ durch Wärmebad}}}
	\end{align*} 	
	
	$\overline{\mathcal{T}}$ : Durchschnitt der Temperatur über das Zeitintervall $\Delta \tau$\\
	Dies führt zu...
\end{frame}

\begin{frame} 
	\frametitle{Reibungskoeefizient $\gamma_i$: Herleitung} 
	
	Dies führt zu...
	\begin{align*}
		\dot{ \overline{\mathcal{T}} } = 2 \gamma [T_0 - \overline{\mathcal{T}}(t)]  .
	\end{align*}
	Vergleiche diese Formel mit 
	\begin{align*}
	\dot{ \overline{\mathcal{T}} } = \zeta_T^{-1} [T_0 - \overline{\mathcal{T}}(t)] 
	\end{align*}
	Also gilt:
	\begin{block}{Wahl von $\gamma$}
		Der Reibungskoefizient $\gamma$ kann für alle Partikel gewählt werden als 
		\begin{align*}
			\gamma = \frac{1}{2 \zeta_T}
		\end{align*}
	\end{block}
\end{frame}


\begin{frame} 
	\frametitle{Eigenschaften der stochastischen Dynamik} 
	\begin{block}{Eigenschaften des SD Algorithmus}
		\begin{itemize}
			\item Trajektorie ist verfügbar und stetig
			\item  Trajektorie ist nicht deterministisch
			\item Bewegungsgleichung ist nicht Zeitreversibel
		\end{itemize}
	\end{block}
\end{frame}
	


\section{Temperatur Relaxatierung mit stochastischer Verknüpfung}
\begin{frame}
	\frametitle{Das System}
	\begin{itemize}
		\item Wärmebad
		\item konstantes Volumen
		\item konstante Anzahl an Partikeln
		\item Ziel: System bei konstanter Temperatur modellieren 
	\end{itemize}
	skizze?
\end{frame}

\begin{frame} 
	\frametitle{Idee des Anderson Thermostats} 
	\begin{itemize}
		\item Idee von Anderson
		\item Partikel des Systems kollidieren
		\item Kollisionen durch neue Geschwindigkeiten der Partikel modelliert
		\item kinetische Energie ändert sich
	\end{itemize}
\end{frame}


\begin{frame} 
	\frametitle{Umsetzung des Anderson Thermostats}
	\begin{itemize}
		\item Modifizierung der Newtonschen Bewegungsgleichung, durch Störung  in jedem Zeitschritt, in dem eine Kollision stattfindet
		\item Wahl des Zeitpunkts der Kollision
		\item Wahl der neuen Geschwindigkeit
	\end{itemize} 
\end{frame}

\begin{frame} 
	\frametitle{Zeitpunkt der Kollision}
	\begin{itemize}
		\item betrachte einzelnen Partikel $i$
		\item $\tau$: Zeitintervall zwischen zwei aufeinanderfolgenden Kollisionen
		\item $\tau$ ist gegeben durch Wahrscheinlichkeitsverteilung
		\begin{align*}
		p(\tau)= \alpha e^{- \alpha \tau}
		\end{align*}
		\item vor Simulation Festlegung der Folge der Zeitintervalle $\tau$ für jeden Partikel
	\end{itemize} 
\end{frame}

\begin{frame} 
	\frametitle{Wahl der neuen Geschwindigkeit}
	 Wahl der Geschwindigkeit mittels Maxwell-Boltzmann Verteilung
	\begin{align*}
		p(\dot{r}_{i \mu}) = {\frac{\beta m_i}{2 \pi}}^{1/2} e^{-1/2 \beta m_i \dot{r}}_{i \mu}
	\end{align*}
	$\beta = (k_B T_0)^{-1}$
\end{frame}


\begin{frame} 
	\frametitle{Newtonsche Bewegungsgleichung für das Anderson Thermostat}
	\begin{align*}
	\ddot{r}_i(t) = m_i^{-1} F_i(t) + \sum\limits_{n=1}^{\infty} \delta \left(  t - \sum\limits_{m=1}^n \tau_{i,m}\right) \left( \dot{r}_{i,n}(t) - \dot{r}_i(t) \right)
	\end{align*}
	 $\{\tau_{i,n}| n=1,2, \dots\}$:  Folge der Neuzuweisung der Geschwindigkeiten für i-tes Partikel \\
	 $\dot{r}_{i,n}(t)$: neue Geschwindigkeit nach dem n-ten Intervall
\end{frame}


\begin{frame} 
	\frametitle{Wahl der Kollisionsfrequenz $\alpha$}
	siehe stochastische Dynamik
	\begin{description}
		\item[$\alpha_i=0 \forall i$:] Molekulare Simulation
		\item[$\alpha_i$ zu klein:] keine Temperaturkontrolle
		\item[$\alpha_i$ zu groß:] Dynamik des Systems gestört 
	\end{description}
	\begin{block}{Wahl von $\alpha$}
		Die Kollisionsfrequenz $\alpha$ kann gewählt werden als
		\begin{align*}
			\alpha = 2/3 (N k_B)^{-1} c_{\nu} \zeta_T^{-1}
		\end{align*}
	\end{block}	
\end{frame}


\begin{frame} 
	\frametitle{Eigenschaften der stochastischen Verknüpfung} 
	\begin{block}{Eigenschaften des SV Algorithmus}
		\begin{itemize}
			\item Trajektorie ist verfügbar und stetig
			\item  Trajektorie ist nicht deterministisch
			\item Bewegungsgleichung ist nicht Zeitreversibel
		\end{itemize}
	\end{block}
\end{frame}


\end{document}