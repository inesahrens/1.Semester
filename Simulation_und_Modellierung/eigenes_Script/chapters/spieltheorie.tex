% ===========
% Spieltehorie
% ===========

\chapter{Spieltheorie}
\label{chap:spieltheorie}

\begin{bsp}[Gefangenendilemma]
	Bankräuber A und B sind verhaftet. Ohne Geständnis kriegen beide 3 Jahre haft, gesteht einer, bekommt er nur 1 Jahr, der andere 9. Falls beide gestehen bekommen beide 7 Jahre Haft
\end{bsp}

\begin{bsp}[Geschlechterkampf]
	Die Frau möchte zum Fußball Spiel, der Mann zum Helene Fischer Konzert. Sie können nicht miteinander kommunizieren und sie haben sich nicht abgesprochen, wo sie sich treffen. Beide wollen sich am liebsten treffen. 
\end{bsp}

\section{Spiele in Strategischer Normalform}

\begin{defi}[Spieler]
	Ein Spieler ist ein Beteiligter in einem Spiel. Er wird mit $X \in M$ bezeichnet, wobei M die Menge aller Spieler ist.  
\end{defi}

\begin{defi}[Strategie]
	Strategien sind die Handlungen, die die Spieler ausführen. Dabei bezeichnet $S_X$ die Menge der Strategien von Spieler X. 
\end{defi}

\begin{thm}
	Falls ein Spiel nur zwei Spieler $X$ und $Y$ hat, gilt 
	\begin{align*}
		S_Y=S_{-X}
	\end{align*}
\end{thm}

\begin{defi}[endliches Spiel]
	ein Spiel heißt endlich, falls jeder Spieler nur endlich viele Strategien hat. Es gilt $n_X=|S_X|$ 
\end{defi}

\begin{defi}[Menge aller Strategiepaare]
	Sei ein Spiel mit zwei Spielern $A$ und $B$ gegeben. Dann ist die Menge aller Strategiepaare $S:=S_A \times S_B$  
\end{defi}

\begin{defi}[Auszahlungsfunktion]
	\begin{align*}
		U_X: S \rightarrow \R
	\end{align*}
	beschreibt den Nutzen. 
\end{defi}

\begin{defi}[Nutzenmatrix]
	Betrachte ein zwei Spieler Spiel mit die Spielern $A$ und $B$. Sei $S_A=\{a_1, \cdots, a_{n_A}\}$ und $S_B=\{b_1, \cdots, b_{n_B}\}$. Dann wird die Nutzenmatrix $U$ definiert durch
	\begin{align*}
		U \in \R^{n_A \times n_B}, \hspace{2ex} U_{ij}:=\left( U_{ij}^A, U_{ij}^B \right), \hspace{2ex} i \in \{ 1,\dots, n_A \} \hspace{1ex} j \in \{ 1, \dots, n_B\}
	\end{align*} 
	wobei $U_{ij}^A:=U_A(a_j,b_j)$ mit der Nutzenfunktion $U_A$.  
\end{defi}
\begin{noti}
	A ist der Zeilenspieler und B ist der Spaltenspieler. 
\end{noti}

\begin{bsp}[Gefangenendilemma]
	Die Spieler sind die Gefangenen A und B. Die Strategien sind Schweigen und Gestehen, also $S_A=S_B=\{S,G\}$. Die Auszahlungsfunktionen werden definiert durch
	\begin{align*}
	& 	U_A:S \rightarrow \R \\
	& 	\begin{array}{ll}
			U_A(S,S)= -3 & U_A(S,G)=-9 \\
			U_A(G,S)= -1 & U_A(G,G)=-7 \\ 
		\end{array}
	\end{align*}
	$U_B$ analog, nur gespiegelt an der Diagonalen. Dadurch ergibt sich eine Nutzenmatrix
	\begin{align*}
		U^{AB}= 
		\begin{pmatrix}
			(-7,-7) & (-1,-9) \\
			(-9,-1) & (-3,-3) 
		\end{pmatrix}
	\end{align*}
\end{bsp}

\begin{defi}[Nullsummenspiel]
	Ein Spiel mit zwei Spieler $X$ und $Y$ heißt Nullsummenspiel, falls gilt 
	\begin{align*}
		U_A(S)=-U_B(S)
	\end{align*}
\end{defi}

\begin{defi}[Strategische Normalform]
	Eine strategische Normalform ist die Darstellung eines Spiels mit Strategiemengen und Auszahlungsfunktion. 
\end{defi}

\begin{defi}[vollständige Information]
	Beiden Spielern ist die komplette Auszahlungsfunktion bekannt. 
\end{defi}

\section{Spiele ohne Annahmen über den Gegner}
Betrachten wir hier zwei Spieler Spiele, in denen B seine Entscheidungen unabhängig vom Spieler A trifft. A ist das bekannt. Es reicht, sich nur die Auszahlungsfunktion $U_A$ anzusehen. 
\begin{bsp}
	A ist eine Person mit oder ohne Regenschirm und B ist das Wetter. 
\end{bsp} 
Es gibt hier zwei Arten von Spielen: Das Spiel mit Gewissheit und das mit Risiko. 

\subsection{Spiel mit Gewissheit}
A kennt die Strategie $b_j$ von Spieler B und maximiert seinen Nutzen. Wähle
\begin{align*}
	\tilde{i} \in \{ 1, \dots n_A \} \text{ mit } U_{\tilde{i}j}=\max\limits_{1 \leq i \leq n_A} U_{ij}
\end{align*}

\subsection{Spiel mit Risiko}
A hat keine Information über die Wahl der Strategie von B. 

\subsubsection{Risikobereiter Spieler}
Zunächst wählt A das eigene Maximum und geht davon aus, dass B auch das Maximum gewählt hat. Das heißt: Wähle 
\begin{align*}
	\tilde{i} \in \{1, \dots, n_A  \} \text{ mit } \max\limits_{1 \leq j \leq n_B} U_{\tilde{i}j}=\max\limits_{1 \leq i \leq n_A} \max\limits_{1 \leq j \leq n_B} U_{ij}
\end{align*} 

\subsubsection{Vorsichtiger Spieler}
Der vorsichtige Spieler probiert den garantierten Gewinn zu maximieren. Das heißt: Wähle 
\begin{align*}
	\tilde{i} \in \{1, \dots, n_A  \} \text{ mit } \min\limits_{1 \leq j \leq n_B} U_{\tilde{i}j}=\max\limits_{1 \leq i \leq n_A} \min\limits_{1 \leq j \leq n_B} U_{ij}
\end{align*} 

\begin{bsp}[Regenschirm?]
	\begin{align*} U=
		\begin{pmatrix}
			0 & 30 \\ 10 & 10
		\end{pmatrix}
	\end{align*}
	Der vorsichtige Spieler wählt die zweite Zeile und der risikofreudige Spieler die erste Zeile. 
\end{bsp}

\section{Reaktionsabbildungen}

