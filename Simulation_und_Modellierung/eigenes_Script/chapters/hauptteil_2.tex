% ========================
% Hauptteil der Arbeit II
% ========================

\chapter{Informationssuche im Netz}

\textbf{Ziel:} Akzeptable Lösung der Problemstellung mit gewünschten Eigenschaften.

\textbf{Information Retrival (IR):}  Großer Datenbestand - Suchanfrage - Ergebnis

\textbf{Grobes Modell einer Internetsuchmaschine:} Zunächst wird im Index gesucht. Die ergbit eine (unsortierte) Trefferliste für "alle" Suchanfragen relevanter Webseiten. Diese Trefferliste wird anhand von Ratingfunktionen (Relevanz/ Qualität) sortiert. 

\textbf{inventierter Index:} Erzeugt eine Liste aller Webseiten zu jedem möglichen Suchbegriff. Für jede Webseite ist die Sichtbarkeit (Suchbegriff im Titel/ im Text,...) des Suchbegriffes aufgeführt. 

\section{Page Rank nach Web-Graphen} 

