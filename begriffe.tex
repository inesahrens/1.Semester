\documentclass[]{article}
\usepackage[utf8]{inputenc}
\usepackage{german}
\usepackage{amsmath}
\usepackage{amssymb}
\usepackage{amsthm}

%opening
\title{wichtige begriffe der Thermodynamik}
\author{Ines Ahrens}

\begin{document}

\maketitle



\section{Begriffe}

Die Kraft ist die Ableitung des Potentials.
Hamiltonglg beschreibt die Energie der Zustände im Phasenraum abhängig von Impuls- und Ortskoordinaten.

$https://www.univie.ac.at/physikwiki/images/8/80/Der_Phasenraum.pdf$

\subsection{Feld}
Referenz: Wiki: Feld(Physik)

Ein Feld beschreibt die räumliche Verteilung einer physikalischen Größe. Beispielsweise kann die räumliche Verteilung der Temperatur einer Herdplatte durch ein Temperaturfeld beschrieben werden oder die räumliche Verteilung der Dichte in einem Körper durch ein Massendichtefeld. In diesem Sinne ist ein Feld ein mathematisches Hilfsmittel, das die eigentlich punktweise definierten physikalischen Eigenschaften eines ausgedehnten oder aus Untersystemen zusammengesetzten Systems in einer Größe, dem Feld, zusammenfasst.

\subsection{Fluss}
Referenz: Wiki: Fluss(Physik) 

Als Fluss werden verschiedene physikalische Größen bezeichnet, die sich als Produkt eines Feldes und einer Fläche ergeben. Beispiele sind magnetische Fluss, elektrischer Fluss, Volumenstrom. 

\subsection{Freiheitsgrad}
Referenz: Wiki: Freiheitsgrad

Als Freiheitsgrad F bzw. f, bei kinematischen Ketten auch Laufgrad, wird die Zahl der voneinander unabhängigen (und in diesem Sinne „frei wählbaren“) Bewegungsmöglichkeiten eines Systems bezeichnet. Die einzelnen Bewegungsmöglichkeiten werden auch Freiheiten genannt. Ein starrer Körper im Raum hat demnach den Freiheitsgrad f = 6, denn man kann den Körper in drei voneinander unabhängige Richtungen bewegen (Translation) und um drei voneinander unabhängige Achsen drehen (Rotation).

In einem etwas anderen Sprachgebrauch wird jede der unabhängigen Bewegungsmöglichkeiten eines Systems, also jede der genannten Freiheiten, als ein Freiheitsgrad bezeichnet. In diesem Sinne hat ein starrer Körper ohne Bindungen drei Translationsfreiheitsgrade und drei Rotationsfreiheitsgrade.
Jedes Molekül mit n Atomen hat allgemein $f = 3n$ Freiheitsgrade, weil man für jedes Atom drei Koordinaten braucht, um seine Position zu definieren. Diese kann man formal in Translations-, Rotations- und innere Schwingungsfreiheitsgrade einteilen:

\begin{align} f & = f_\mathrm{trans} + f_\mathrm{rot} + f_\mathrm{vib}\\ 
\Rightarrow f_\mathrm{vib} & = 3n - f_\mathrm{trans} - f_\mathrm{rot} 
\end{align}

Komplexe Moleküle mit vielen Atomen haben daher viele Schwingungsfreiheitsgrade (siehe Molekülschwingung) und liefern somit einen hohen Beitrag zur Entropie.

\subsection{innere energie}
Referenz: Wiki: Innere Energie

Die innere Energie U ist die gesamte für thermodynamische Umwandlungsprozesse zur Verfügung stehende Energie eines physikalischen Systems, das sich in Ruhe und im thermodynamischen Gleichgewicht befindet. Die innere Energie setzt sich aus einer Vielzahl anderer Energieformen zusammen, sie ist nach dem ersten Hauptsatz der Thermodynamik in einem abgeschlossenen System konstant.

\subsection{makrozustand}
Referenz: Wiki: Makrozustand

Ein Makrozustand beschreibt in der Thermodynamik und statistischen Physik ein System mit vielen Freiheitsgraden (also z. B. ein Gas, das aus 1 mol ca. $10^{23}$ Einzelteilchen besteht) durch einige wenige Zustandsvariablen, wie Energie, Temperatur, Volumen, Druck, chemische Zusammensetzung oder Magnetisierung.

In der Mechanik lässt sich ein System aus N Teilchen vollständig beschreiben, indem man jedem Teilchen einen Orts- und Geschwindigkeitsvektor zuordnet. Man spricht hier von einem Mikrozustand. Dieser kann durch einen Punkt im Phasenraum dargestellt werden.

Für viele Teilchen (Teilchenzahl$ N \sim 10^{23}$) ist es jedoch praktisch unmöglich, einen mikroskopischen Anfangszustand zu bestimmen oder die Bewegungsgleichung für das System zu lösen. In chaotischen Systemen ist die Bestimmung der Bahn des Systems auch prinzipiell unmöglich, da kleinste Änderungen der Anfangsbedingungen zu beliebig großen Abweichungen führen.

Die mikroskopische Lösung der Bewegungsgleichung ist aber auch gar nicht notwendig, da die makroskopischen Eigenschaften nur von wenigen Parametern abhängen.

Zu einem bestimmten Makrozustand, der durch wenige makroskopischen Zustandsvariablen festgelegt ist, sind sehr viele Mikrozustände möglich. Diese bilden eine kontinuierlich verteilte Gesamtheit im Phasenraum. Der Makrozustand wird also durch ein statistisches Konzept bestimmt (Wahrscheinlichkeitsverteilung der Mikrozustände). Die Schwankungen der makroskopischen Größen werden aufgrund der hohen Teilchenzahlen vernachlässigbar gering.

Mit makroskopischen Größen kann man so makroskopische, deterministische Gesetze aufstellen. Kennt man z.B. für ein Gas die makroskopischen Zustandsgrößen Volumen, Temperatur und Teilchenzahl, so lässt sich der Druck eindeutig berechnen (Thermische Zustandsgleichung idealer Gase).

\subsection{intensive und extensive Größe}
Refernz: Wiki: Intensive Größe

Eine intensive Größe ist eine Zustandsgröße, die sich bei unterschiedlicher Größe des betrachteten Systems nicht ändert.

Eine extensive Größe ist eine Zustandsgröße, die sich mit der Größe des betrachteten Systems ändert. Beispiele hierfür sind Masse, Stoffmenge, Volumen, Entropie sowie die thermodynamischen Potentiale (innere Energie, freie Energie, Enthalpie und freie Enthalpie). 

\subsection{Entropie}
Referenz: Wiki: Entropie

Die Entropie ist eine fundamentale thermodynamische Zustandsgröße mit der SI-Einheit Joule pro Kelvin, also J/K.

Die in einem System vorhandene Entropie ändert sich bei Aufnahme oder Abgabe von Wärme. In einem abgeschlossenen System, bei dem es keinen Wärme- oder Materieaustausch mit der Umgebung gibt, kann die Entropie nach dem zweiten Hauptsatz der Thermodynamik nicht abnehmen. Mit anderen Worten: Entropie kann nicht vernichtet werden. Es kann im System jedoch Entropie entstehen. Prozesse, bei denen dies geschieht, werden als irreversibel bezeichnet, d. h. sie sind – ohne äußeres Zutun – unumkehrbar. Entropie entsteht z. B. dadurch, dass mechanische Energie durch Reibung in thermische Energie umgewandelt wird. Da die Umkehrung dieses Prozesses nicht möglich ist, spricht man auch von einer „Energieentwertung“.

In der statistischen Mechanik stellt die Entropie eines Makrozustands ein Maß für die Zahl der zugänglichen, energetisch gleichwertigen Mikrozustände dar. Makrozustände höherer Entropie haben mehr Mikrozustände und sind daher statistisch wahrscheinlicher als Zustände niedrigerer Entropie. Folglich bewirken die inneren Vorgänge in einem sich selbst überlassenen System im statistischen Mittel eine Annäherung an den Makrozustand, der bei gleicher Energie die höchstmögliche Entropie hat. Da in einem anfänglich gut geordneten System durch innere Prozesse die Ordnung nur abnehmen kann, wird diese Interpretation des Entropiebegriffs umgangssprachlich häufig dadurch umschrieben, dass Entropie ein „Maß für Unordnung“ sei. Allerdings ist Unordnung kein physikalischer Begriff und hat daher auch kein physikalisches Maß. Besser ist es, die Entropie als ein „Maß für die Unkenntnis des atomaren Zustands“ zu begreifen, obwohl auch Unkenntnis kein physikalisch definierter Begriff ist

\subsection{Thermodynamik}
Referenz: Wiki: Thermodynamik

Die Thermodynamik beschäftigt sich mit der Möglichkeit, durch Umverteilen von Energie zwischen ihren verschiedenen Erscheinungsformen Arbeit zu verrichten. Die Grundlagen der Thermodynamik wurden aus dem Studium der Volumen-, Druck-, Temperaturverhältnisse bei Dampfmaschinen entwickelt.

Man unterscheidet zwischen offenen, geschlossenen und abgeschlossenen (isolierten) thermodynamischen Systemen. Bei einem offenen System fließt – im Gegensatz zum geschlossenen – Materie über die Systemgrenze, abgeschlossene Systeme sind auch Energiedicht. Nach dem Energieerhaltungssatz bleibt darin die Summe aller Energieformen (thermische, chemische, Federspannung, Magnetisierung usw.) konstant.

Die Thermodynamik bringt die Prozessgrößen Wärme und Arbeit an der Systemgrenze mit den Zustandsgrößen in Zusammenhang, welche den Zustand des Systems beschreiben. Dabei wird zwischen intensiven Zustandsgrößen (beispielsweise Temperatur T, Druck p, Konzentration n und chemisches Potential $\mu$) und extensiven Zustandsgrößen (beispielsweise innere Energie U, Entropie S, Volumen V und Teilchenzahl N) unterschieden.

Auf der Basis von vier fundamentalen Hauptsätzen sowie materialspezifischen, empirischen Zustandsgleichungen zwischen den Zustandsgrößen (siehe z. B. Gasgesetz) erlaubt die Thermodynamik durch die Aufstellung von Gleichgewichtsbedingungen Aussagen darüber, welche Änderungen an einem System möglich sind (beispielsweise welche chemischen Reaktionen oder Phasenübergänge ablaufen können, aber nicht wie) und welche Werte der intensiven Zustandsgrößen dafür erforderlich sind. Sie dient zur Berechnung von frei werdender Wärmeenergie, von Druck-, Temperatur- oder Volumenänderungen, und hat daher große Bedeutung für das Verständnis und die Planung von Prozessen in Chemieanlagen, bei Wärmekraftmaschinen sowie in der Heizungs- und Klimatechnik.


\subsection{thermodynamisches Potential}
referenz: Wiki: Thermodynamisches Potential

Thermodynamische Potentiale sind in der Thermodynamik Größen, die von ihrem Informationsgehalt her das Verhalten eines thermodynamischen Systems im Gleichgewicht vollständig beschreiben. Sie entsprechen vom Informationsgehalt der inneren Energie U,[1] deren natürliche Variablen S,V,N alle extensiv sind (Fundamentalgleichung).

Thermodynamische Potentiale, die Energien sind, lassen sich durch Legendre-Transformation aus der inneren Energie U(S,V,N) herleiten, haben jedoch anders als diese eine oder mehrere intensive Größen als natürliche Variablen (T, p, $\mu$). Die intensiven Größen entstehen bei der Koordinatentransformation als Ableitungen der inneren Energie nach ihren extensiven Variablen.

Daneben gibt es weitere thermodynamische Potentiale, die keine Energien sind, beispielsweise die Entropie S(U,V,N).


\subsection{Thermodynamisches System}
Referenz: Wiki: Thermodynamisches System

Ein thermodynamisches System ist ein räumlich eingegrenzt betrachtetes physikalisches System, für das eine Bilanzgleichung wie eine Energiebilanz oder, bei offenen Systemen, eine Stoffbilanz aufgestellt werden kann. Beim geschlossenen System werden nur die Energien (Wärme und Arbeit) betrachtet, die über die Systemgrenze fließen und dadurch mit der Änderung der inneren Energie den Zustand des Systems verändern. Bei einem isolierten System findet keinerlei Austausch mit der Umgebung statt.


\subsection{thermodynamisches Gleichgewicht}
Referenz: Wiki: Gleichgewicht(Physik $->$ Thermodynamik) 

Allgemein ist ein Gleichgewicht die Ausgeglichenheit aller Potentiale und Flüsse in einem System.
 
Ein System ist im thermodynamischen Gleichgewicht, wenn es in einem stationären Zustand ist, in dem alle makroskopischen Flüsse verschwinden. Dann gilt grundsätzlich das Kräftegleichgewicht aus Gibbs freier Enthalpie. Das heißt, dass die Thermodynamischen Potenziale kein Gefälle haben, die eine Änderung der Potentialgrößen des Systems antreiben.


\subsection{Ensemble}
Referenz: Wiki: Ensemble(Physik) 

Ein Ensemble oder eine Gesamtheit ist in der statistischen Physik eine Menge gleichartig präparierter Systeme von Teilchen im thermodynamischen Gleichgewicht. 

\subsection{kanonisches Ensemble}
Referenz: Wiki: kanonisches Ensemble

Das kanonische Ensemble (auch NVT-Ensemble oder Gibbs-Ensemble nach J. W. Gibbs) ist in der statistischen Physik ein System mit festgelegter Teilchenzahl in einem konstanten Volumen, das Energie mit einem Reservoir austauschen kann und mit diesem im thermischen Gleichgewicht ist. Dies entspricht einem System mit vorgegebener Temperatur, wie ein geschlossenes System (kein Teilchenaustausch) in einem Wärmebad (makroskopisches System, das sehr viel größer ist als das betrachtete System).

\subsection{mikrokanonisches ensemble}
Referenz: Wiki: mikrokanonisches Ensemble

Das mikrokanonische Ensemble beschreibt in der statistischen Physik ein System mit festgelegter Gesamtenergie im thermodynamischen Gleichgewicht. Es unterscheidet sich damit z. B. vom kanonischen Ensemble, in dem ein thermischer Kontakt mit der Umgebung besteht, der bei festgelegter Temperatur eine fluktuierende Gesamtenergie erlaubt.


\end{document}
